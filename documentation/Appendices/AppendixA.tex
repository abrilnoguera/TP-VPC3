% Appendix A
\chapter{Planificación del Proyecto}
\label{AppendixA}

A continuación se presenta la tabla detallada de planificación del proyecto, que incluye todas las tareas necesarias para completar el trabajo desde la inicialización del repositorio hasta la presentación final.

\begin{table}[h]
\centering
\caption{Planificación del Proyecto}
\label{tab:planificacion1}

\begin{adjustbox}{max width=\textwidth}
\footnotesize
\begin{tabular}{p{3.3cm} p{9cm} p{1.7cm} p{1.5cm}}
\toprule
\textbf{Título} & \textbf{Descripción} & \textbf{Responsable} & \textbf{Estado} \\
\midrule

Inicializar cookiecutter & Ejecutar \texttt{cookiecutter-data-science} para generar la estructura base del proyecto. & Abril & Completado \\
Configurar entorno Conda & Crear entorno \texttt{product\_tagger} e instalar dependencias. & Abril & Completado \\
Crear repositorio GitHub & Crear repositorio, subir estructura inicial, conectar con VSCode. & Abril & Completado \\
Definir proyecto final & Acordar que el proyecto será un sistema Product Tagger basado en Vision Transformers. & Equipo & Completado \\
Definir Business Case & Documentar problema de negocio, costos y beneficios. & Abril & Completado \\
Obtención de Datos & Implementar descarga automática del dataset desde \texttt{KaggleHub}. & Abril & Completado \\
EDA & Explorar \texttt{styles.csv}, distribución de atributos y ejemplos de imágenes. & Abril & Completado \\
Train/Val/Test Split & Crear particiones estratificadas & Abril & Completado \\
Preparación imágenes & Redimensionar y normalizar imágenes para ViT. & Abril & Completado \\
Implementar PyTorch Dataset & Construir clase Dataset para imágenes y etiquetas. & XX & Pendiente \\
Implementar augmentations & Agregar normalización, resize, flips y transforms. & XX & Pendiente \\
Cargar Vision Transformer & Cargar ViT/DeiT preentrenado y adaptar classifier. & XX & Pendiente \\
Pipeline de entrenamiento & Entrenar ViT con optimizer, scheduler y early stopping. & XX & Pendiente \\
Hostear MLflow para experimentos & Configurar MLflow para registrar experimentos y métricas del modelo. & Pedro & En curso \\
Registrar métricas & Guardar loss, accuracy, F1 y mAP por época. & XX & Pendiente \\
Evaluar modelo & Evaluar rendimiento sobre test set. & XX & Pendiente \\
Visualizar resultados & Graficar curvas, métricas y matriz de confusión. & XX & Pendiente \\
Ejemplos de predicción & Mostrar aciertos y errores. & XX & Pendiente \\
Guardar modelo entrenado & Exportar modelo final a \texttt{models/}. & XX & Pendiente \\
Escribir README & Instrucciones de instalación, entrenamiento e inferencia. & XX & Pendiente \\
Redactar Objetivo & Escribir objetivo del proyecto en el informe. & XX & Pendiente \\
Redactar Arquitectura & Diagramar y explicar el pipeline. & XX & Pendiente \\
Redactar Implementación técnica & Detallar módulos y diseño del sistema. & XX & Pendiente \\
Redactar Evaluación & Explicar métricas y análisis de resultados. & XX & Pendiente \\
Redactar Resultados & Incluir visualizaciones y análisis final. & XX & Pendiente \\
Redactar Conclusiones & Limitaciones y líneas futuras. & XX & Pendiente \\
Preparar presentación & Diapositivas y narrativa final. & XX & Pendiente \\
Practicar presentación & Ensayo del equipo. & XX & Pendiente \\
Entrega final código & Limpiar repo y hacer tagging final. & XX & Pendiente \\
Entrega final PDF & Compilar y entregar memoria final. & XX & Pendiente \\

\bottomrule
\end{tabular}
\end{adjustbox}
\end{table}
