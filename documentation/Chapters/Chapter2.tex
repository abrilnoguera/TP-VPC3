\chapter{Análisis Exploratorio de Datos} % Main chapter title

\label{Chapter2}

El presente capítulo describe el análisis exploratorio (EDA) realizado sobre el conjunto de datos utilizado para el entrenamiento del sistema de etiquetado automático de productos. El objetivo del EDA es comprender la estructura del dataset, evaluar la calidad de las imágenes, identificar patrones relevantes y detectar posibles problemas que condicionen las decisiones de preprocesamiento y modelado. A partir de este análisis se establecen los requerimientos y transformaciones necesarias para garantizar un entrenamiento adecuado del modelo basado en \textit{Vision Transformers}.

\section{Descripción general del dataset}

El proyecto emplea el dataset público \textit{Fashion Product Images (Small)}, disponible en Kaggle, que contiene más de 44.000 imágenes de productos de moda junto con metadatos tabulares provistos en el archivo \texttt{styles.csv}. Cada imagen está asociada a un identificador único (\texttt{id}) y cuenta con atributos descriptivos como:

\begin{itemize}
\item \textbf{masterCategory}: categoría general del artículo.
\item \textbf{subCategory}: categoría específica intermedia.
\item \textbf{articleType}: tipo concreto de producto.
\item \textbf{gender}: público objetivo.
\item \textbf{baseColour}, \textbf{season}, \textbf{usage}, entre otros.
\end{itemize}

Durante el EDA se realizó una verificación de integridad del archivo tabular y una validación de la existencia y apertura de cada imagen asociada.

\section{Análisis de valores faltantes}

La figura~\ref{fig:missing_values} muestra la proporción de valores faltantes por atributo. Las variables con mayor incompletitud corresponden a \texttt{usage}, \texttt{season} y \texttt{baseColour}, ninguna de las cuales forma parte de los atributos objetivo del proyecto. Los campos esenciales para la clasificación (\texttt{gender}, \texttt{masterCategory}, \texttt{subCategory} y \texttt{articleType}) no presentan valores faltantes.

\begin{figure}[H]
\centering
\includegraphics[width=0.75\linewidth]{figures/missing_values.png}
\caption{Proporción de valores faltantes en el archivo \texttt{styles.csv}.}
\label{fig:missing_values}
\end{figure}

\textbf{Conclusión:} No es necesario realizar imputación compleja; el dataset es adecuado para construir las etiquetas objetivo sin modificaciones adicionales.

\section{Distribución de clases}

La figura~\ref{fig:class_articleType} ilustra la distribución de clases para \texttt{articleType}. Se observa un marcado desbalance, con categorías como \textit{Tshirts}, \textit{Shirts} y \textit{Casual Shoes} concentrando una cantidad significativamente mayor de muestras respecto de categorías minoritarias.

\begin{figure}[H]
\centering
\includegraphics[width=\linewidth]{figures/class_distribution_articleType.png}
\caption{Distribución de clases para \texttt{articleType}.}
\label{fig:class_articleType}
\end{figure}

Patrones similares se observan en \texttt{subCategory} y \texttt{masterCategory} (figuras~\ref{fig:class_subCategory} y~\ref{fig:class_masterCategory}), donde predominan \textit{Topwear}, \textit{Shoes} y \textit{Apparel}, respectivamente.

\begin{figure}[H]
\centering
\includegraphics[width=\linewidth]{figures/class_distribution_subCategory.png}
\caption{Distribución de clases para \texttt{subCategory}.}
\label{fig:class_subCategory}
\end{figure}

\begin{figure}[H]
\centering
\includegraphics[width=\linewidth]{figures/class_distribution_masterCategory.png}
\caption{Distribución de clases para \texttt{masterCategory}.}
\label{fig:class_masterCategory}
\end{figure}

En cuanto al atributo \texttt{gender}, la distribución está dominada por \textit{Men} y \textit{Women}, mientras que \textit{Boys}, \textit{Girls} y \textit{Unisex} representan una proporción menor (figura~\ref{fig:class_gender}).

\begin{figure}[H]
\centering
\includegraphics[width=\linewidth]{figures/class_distribution_gender.png}
\caption{Distribución de clases para \texttt{gender}.}
\label{fig:class_gender}
\end{figure}

\textbf{Conclusión:} El dataset presenta un fuerte desbalance de clases que puede afectar el aprendizaje del modelo. Esto justifica la aplicación posterior de estrategias como \textit{class weighting}, muestreo estratificado o técnicas de reponderación durante el entrenamiento.

\section{Inspección visual de muestras}

La figura~\ref{fig:random_samples} muestra una selección aleatoria de imágenes del dataset. Se observa que la mayoría de los productos están fotografiados sobre fondo blanco, correctamente centrados y con iluminación homogénea. La ausencia de fondos complejos y ruido visual simplifica el aprendizaje del modelo.

\begin{figure}[H]
\centering
\includegraphics[width=\linewidth]{figures/random_samples.png}
\caption{Muestras aleatorias de imágenes del dataset.}
\label{fig:random_samples}
\end{figure}

\textbf{Conclusión:} Las condiciones controladas de captura favorecen un desempeño estable en modelos de visión profunda.

\section{Variabilidad en tamaños de imagen}

El análisis de resolución reveló que todas las imágenes tienen un tamaño uniforme de \textbf{60×80 px}, con desviación estándar nula. Si bien esta consistencia elimina la necesidad de correcciones geométricas, la resolución es insuficiente para arquitecturas modernas como los \textit{Vision Transformers}, que operan con tamaños estándar de 224×224 px.

\textbf{Conclusión:} Es necesario reescalar todas las imágenes a 224×224 px para compatibilidad con el modelo seleccionado.

\section{Análisis de calidad de imagen}

Se evaluaron métricas complementarias de calidad:

\begin{itemize}
\item \textbf{Brillo}: valores centrados entre 200 y 230, indicando imágenes bien iluminadas.
\item \textbf{Contraste}: niveles moderados, consistentes entre productos.
\item \textbf{Nitidez}: varianza Laplaciana mayormente entre 500 y 4000, suficiente para distinguir bordes.
\item \textbf{Distribución RGB}: canales balanceados sin dominancia cromática.
\end{itemize}

Estas mediciones se ilustran en la figura~\ref{fig:image_quality}.

\begin{figure}[H]
\centering
\includegraphics[width=\linewidth]{figures/image_quality.png}
\caption{Análisis de brillo, contraste, nitidez y distribución de color.}
\label{fig:image_quality}
\end{figure}

\textbf{Conclusión:} La calidad general del dataset es elevada y consistente, lo cual minimiza la necesidad de correcciones avanzadas o normalizaciones fuera del estándar habitual de redes profundas.

\section{Conclusiones del EDA y su impacto en el modelado}

A partir del análisis realizado, se identifican las siguientes decisiones críticas para el pipeline de modelado:

\begin{itemize}
\item \textbf{Redimensionamiento obligatorio}: todas las imágenes deben ser escaladas a 224×224 px para ser compatibles con \textit{Vision Transformers}.
\item \textbf{Normalización estándar}: dada la estabilidad de brillo y color, basta con aplicar la normalización típica de modelos preentrenados (mean/std de ImageNet).
\item \textbf{Augmentations moderados}: la homogeneidad del dataset sugiere utilizar aumentos suaves (flips, jitter ligero) para mejorar generalización sin distorsionar objetos.
\item \textbf{Manejo de desbalance}: será necesario emplear \textit{class weighting}, muestreo estratificado o pérdidas alternativas para evitar que el modelo favorezca categorías dominantes.
\item \textbf{Selección del modelo}: la baja resolución original y la estructura limpia de producto sobre fondo blanco justifican el uso de \textit{Vision Transformers}, que aprovechan patrones espaciales uniformes y generalizan bien ante variaciones mínimas.
\end{itemize}

En conjunto, el EDA confirma que el dataset es adecuado para entrenar un sistema de clasificación de atributos visuales basado en arquitecturas modernas de visión profunda, requiriendo únicamente ajustes controlados para alcanzar condiciones óptimas de entrenamiento.